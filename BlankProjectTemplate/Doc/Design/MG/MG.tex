\documentclass[12pt, titlepage]{article}

\usepackage{fullpage}
\usepackage[round]{natbib}
\usepackage{multirow}
\usepackage{booktabs}
\usepackage{tabularx}
\usepackage{graphicx}
\usepackage{float}
\usepackage{hyperref}
\hypersetup{
    colorlinks,
    citecolor=black,
    filecolor=black,
    linkcolor=red,
    urlcolor=blue
}
\usepackage[round]{natbib}

\newcounter{acnum}
\newcommand{\actheacnum}{AC\theacnum}
\newcommand{\acref}[1]{AC\ref{#1}}

\newcounter{ucnum}
\newcommand{\uctheucnum}{UC\theucnum}
\newcommand{\uref}[1]{UC\ref{#1}}

\newcounter{mnum}
\newcommand{\mthemnum}{M\themnum}
\newcommand{\mref}[1]{M\ref{#1}}

\title{SE 3XA3: Module Guide\\MovieGuide Application}

\author{Team 35, PGH Software Solutions
		\\ Pratyush Bhandari, bhandarp
		\\ Gazenfar Syed, syedg1
		\\ Hamid Ghasemi, ghasemih
}

\date{\today}



\begin{document}

\maketitle

\pagenumbering{roman}
\tableofcontents

\newpage

\listoftables
\listoffigures

\begin{table}[bp]
\caption{\bf Revision History}
\begin{tabularx}{\textwidth}{p{3cm}p{2cm}X}
\toprule {\bf Date} & {\bf Version} & {\bf Notes}\\
\midrule
Nov 8 & 1.0 & Parts 1 and 2\\
Date 2 & 1.1 & Notes\\
\bottomrule
\end{tabularx}
\end{table}

\pagenumbering{arabic}

 
\newpage
\section{Introduction}

\subsection{Overview}
The MovieGuide project is a re-implementation of an open-source movie reviews Android application which allows users to view ratings, description, and trailers for any movie within the API database.

\subsection{Context}
This is the Module Guide Design Documentation for the MovieGuide re-implementation. Readers of this document include, but are not limited to:

\begin{itemize}
	\item New project members: This document gives an insight into the design structure of the project. It contains information about the module hierarchy along with the modular decomposition, so that new members can quickly learn how each module is integrated into the program.
	\item Maintainers: This document also outlines areas where changes are anticipated which will be helpful to the maintainers as they will be able to quickly identify areas of the application that need to be improved. Moreover, maintainers can utilize the breakdown of the modules to better understand the project structure so that they know which modules to update.
	\item Designers/Developers: This module guide describes the design choices that were made when developing the program. It contains an overview of the relationships between all modules which will aid the designers and developers in determining whether the product design is satisfiable for the software requirements.
\end{itemize}

\subsection{Design Principles}
The design principles utilized for our design choices include information hiding and encapsulation. Information hiding keeps certain aspects of the code a secret from the other modules which reduces the complexity of the program and protects it from extensive modifications that could potentially harm the implementation. Encapsulation combines data and methods allowing internal data to be accessed by public methods. Furthermore, modules were implemented with the principles of high cohesion and low coupling to ensure that the modules are related but not heavily dependent on each other.

\subsection{Document Structure}
This document is organized as specified below:
\begin{itemize}
	\item Section 2: anticipated changes and unlikely changes for the current implementation
	\item Section 3: outlines the module hierarchy 
	\item Section 4: summarizes the relationships between modules and software requirements
	\item Section 5: explains the secret and service/responsibility of each module
	\item Section 6: relates modules to requirements and anticipated changes 
	\item Section 7: describes the uses relations between modules
\end{itemize}

\section{Anticipated and Unlikely Changes} \label{SecChange}
\subsection{Anticipated Changes} \label{SecAchange}
\begin{itemize}
	\item AC1: The API used to retrieve information about movies 
	\item AC2: The format of the input data
	\item AC3: The format of the output data
	\item AC4: The User Interface utilized to navigate through the application
	\item AC5: How movie data is stored locally on the application (ie. improving performance)
\end{itemize}

\subsection{Unlikely Changes} \label{SecUchange}
\begin{itemize}
	\item UC1: The input/output devices (system is designed to run on Android devices) 
	\item UC2: The purpose of the software to provide ratings and descriptions about movies
	\item UC3: The storage method of output data (system assumes there is sufficient storage space on the device, so output data will be stored locally)
\end{itemize}

\section{Module Hierarchy} \label{SecMH}
This section provides an overview of the module design. Modules are summarized
in a hierarchy decomposed by secrets in Table \ref{TblMH}. The modules listed
below, which are leaves in the hierarchy tree, are the modules that will
actually be implemented.

\begin{description}
\item [\refstepcounter{mnum} \mthemnum \label{mHH}:] Hardware-Hiding Module
\item ...
\end{description}


\begin{table}[h!]
\centering
\begin{tabular}{p{0.3\textwidth} p{0.6\textwidth}}
\toprule
\textbf{Level 1} & \textbf{Level 2}\\
\midrule

{Hardware-Hiding Module} & ~ \\
\midrule

\multirow{7}{0.3\textwidth}{Behaviour-Hiding Module} & ?\\
& ?\\
& ?\\
& ?\\
& ?\\
& ?\\
& ?\\ 
& ?\\
\midrule

\multirow{3}{0.3\textwidth}{Software Decision Module} & {?}\\
& ?\\
& ?\\
\bottomrule

\end{tabular}
\caption{Module Hierarchy}
\label{TblMH}
\end{table}

\section{Connection Between Requirements and Design} \label{SecConnection}

The design of the system is intended to satisfy the requirements developed in
the SRS. In this stage, the system is decomposed into modules. The connection
between requirements and modules is listed in Table \ref{TblRT}.

\section{Module Decomposition} \label{SecMD}

Modules are decomposed according to the principle of ``information hiding''
proposed by \citet{ParnasEtAl1984}. The \emph{Secrets} field in a module
decomposition is a brief statement of the design decision hidden by the
module. The \emph{Services} field specifies \emph{what} the module will do
without documenting \emph{how} to do it. For each module, a suggestion for the
implementing software is given under the \emph{Implemented By} title. If the
entry is \emph{OS}, this means that the module is provided by the operating
system or by standard programming language libraries.  Also indicate if the
module will be implemented specifically for the software.

Only the leaf modules in the
hierarchy have to be implemented. If a dash (\emph{--}) is shown, this means
that the module is not a leaf and will not have to be implemented. Whether or
not this module is implemented depends on the programming language
selected.

\subsection{Hardware Hiding Modules (\mref{mHH})}

\begin{description}
\item[Secrets:]The data structure and algorithm used to implement the virtual
  hardware.
\item[Services:]Serves as a virtual hardware used by the rest of the
  system. This module provides the interface between the hardware and the
  software. So, the system can use it to display outputs or to accept inputs.
\item[Implemented By:] OS
\end{description}

\subsection{Behaviour-Hiding Module}

\begin{description}
\item[Secrets:]The contents of the required behaviours.
\item[Services:]Includes programs that provide externally visible behaviour of
  the system as specified in the software requirements specification (SRS)
  documents. This module serves as a communication layer between the
  hardware-hiding module and the software decision module. The programs in this
  module will need to change if there are changes in the SRS.
\item[Implemented By:] --
\end{description}

\subsubsection{Input Format Module (\mref{mInput})}

\begin{description}
\item[Secrets:]The format and structure of the input data.
\item[Services:]Converts the input data into the data structure used by the
  input parameters module.
\item[Implemented By:] [Your Program Name Here]
\end{description}

\subsubsection{Etc.}


\subsection{Software Decision Module}

\begin{description}
\item[Secrets:] The design decision based on mathematical theorems, physical
  facts, or programming considerations. The secrets of this module are
  \emph{not} described in the SRS.
\item[Services:] Includes data structure and algorithms used in the system that
  do not provide direct interaction with the user. 
  % Changes in these modules are more likely to be motivated by a desire to
  % improve performance than by externally imposed changes.
\item[Implemented By:] --
\end{description}

\subsubsection{Etc.}

\section{Traceability Matrix} \label{SecTM}

This section shows two traceability matrices: between the modules and the
requirements and between the modules and the anticipated changes.

% the table should use mref, the requirements should be named, use something
% like fref
\begin{table}[H]
\centering
\begin{tabular}{p{0.2\textwidth} p{0.6\textwidth}}
\toprule
\textbf{Req.} & \textbf{Modules}\\
\midrule
R1 & \mref{mHH}, \mref{mInput}, \mref{mParams}, \mref{mControl}\\
R2 & \mref{mInput}, \mref{mParams}\\
R3 & \mref{mVerify}\\
R4 & \mref{mOutput}, \mref{mControl}\\
R5 & \mref{mOutput}, \mref{mODEs}, \mref{mControl}, \mref{mSeqDS}, \mref{mSolver}, \mref{mPlot}\\
R6 & \mref{mOutput}, \mref{mODEs}, \mref{mControl}, \mref{mSeqDS}, \mref{mSolver}, \mref{mPlot}\\
R7 & \mref{mOutput}, \mref{mEnergy}, \mref{mControl}, \mref{mSeqDS}, \mref{mPlot}\\
R8 & \mref{mOutput}, \mref{mEnergy}, \mref{mControl}, \mref{mSeqDS}, \mref{mPlot}\\
R9 & \mref{mVerifyOut}\\
R10 & \mref{mOutput}, \mref{mODEs}, \mref{mControl}\\
R11 & \mref{mOutput}, \mref{mODEs}, \mref{mEnergy}, \mref{mControl}\\
\bottomrule
\end{tabular}
\caption{Trace Between Requirements and Modules}
\label{TblRT}
\end{table}

\begin{table}[H]
\centering
\begin{tabular}{p{0.2\textwidth} p{0.6\textwidth}}
\toprule
\textbf{AC} & \textbf{Modules}\\
\midrule
\acref{acHardware} & \mref{mHH}\\
\acref{acInput} & \mref{mInput}\\
\acref{acParams} & \mref{mParams}\\
\acref{acVerify} & \mref{mVerify}\\
\acref{acOutput} & \mref{mOutput}\\
\acref{acVerifyOut} & \mref{mVerifyOut}\\
\acref{acODEs} & \mref{mODEs}\\
\acref{acEnergy} & \mref{mEnergy}\\
\acref{acControl} & \mref{mControl}\\
\acref{acSeqDS} & \mref{mSeqDS}\\
\acref{acSolver} & \mref{mSolver}\\
\acref{acPlot} & \mref{mPlot}\\
\bottomrule
\end{tabular}
\caption{Trace Between Anticipated Changes and Modules}
\label{TblACT}
\end{table}

\section{Use Hierarchy Between Modules} \label{SecUse}

In this section, the uses hierarchy between modules is
provided. \citet{Parnas1978} said of two programs A and B that A {\em uses} B if
correct execution of B may be necessary for A to complete the task described in
its specification. That is, A {\em uses} B if there exist situations in which
the correct functioning of A depends upon the availability of a correct
implementation of B.  Figure \ref{FigUH} illustrates the use relation between
the modules. It can be seen that the graph is a directed acyclic graph
(DAG). Each level of the hierarchy offers a testable and usable subset of the
system, and modules in the higher level of the hierarchy are essentially simpler
because they use modules from the lower levels.

\begin{figure}[H]
\centering
%\includegraphics[width=0.7\textwidth]{UsesHierarchy.png}
\caption{Use hierarchy among modules}
\label{FigUH}
\end{figure}

\end{document}