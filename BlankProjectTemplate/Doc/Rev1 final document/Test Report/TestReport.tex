\documentclass[12pt, titlepage]{article}

\usepackage{booktabs}
\usepackage{tabularx}
\usepackage{hyperref}
\hypersetup{
    colorlinks,
    citecolor=black,
    filecolor=black,
    linkcolor=red,
    urlcolor=blue
}
\usepackage[round]{natbib}

\title{SE 3XA3: Test Report\\Title of Project}

\author{Team 35, PGH
		\\ Hamid Ghasemi and ghasemih
		\\ Pratyush Bhandari, bhandarp
		\\ Gazenfar Syed, syedg1
}

\date{\today}


\begin{document}

\maketitle

\pagenumbering{roman}
\tableofcontents
\listoftables
\listoffigures

\begin{table}[bp]
\caption{\bf Revision History}
\begin{tabularx}{\textwidth}{p{3cm}p{2cm}X}
\toprule {\bf Date} & {\bf Version} & {\bf Notes}\\
\midrule
Dec3 & 1.0 & Notes\\
Dec4 & 1.1 & Notes\\
\bottomrule
\end{tabularx}
\end{table}

\newpage

\pagenumbering{arabic}

This document ...

\section{Functional Requirements Evaluation}
The aim of these tests is to make sure that user is able to use the software according to the given requirements. These test will inclode, sorting testing, searching teasting, accessing the summary, rate and date released, and trailer testing.\\

Test Name: FRSR \\
Results: The user is able to sort movies based on rate, popularity, and released date. \\ 

Test Name: FRSE \\
Results: The user is able to search a movie by using movie name. \\ 

Test Name: FRAC \\
Results: The user is able to access movie's summary, rate and released date.\\ 

Test Name: FRTR\\
Results: The user is able to watch a movie trailer by simply selecting movie's trailer. \\ \\

\section{Nonfunctional Requirements Evaluation}

\subsection{Usability}
\subsubsection{GUI Testing}
The Graphical User Interface (GUI) was tested by 7 students from Mac who are not in Software Engineering but they were interested to see our apps (From other programs) to reflect the technological experience of the potential users for this program. All the participants observed the timing it took them to perform the requested tasks, and they see the difficulty of the softwar, and how long would take them to get their hands around the app. At the end of the section they all gave feedbacks about our apps and evaluate the difficlty and performance of our application.\\

	
Test Name: SS-1 \\
Results: All participants were able to successfully complete installing the program to their personal phone (they all had android phone) within 2-3 minutes.\\

Test Name: SS-2 \\
Results: All participants were able to successfully complete the task of searching a favorite movie by inputing the movie's name .\\

Test Name: SS-3 \\
Results: All participants were able to successfully the task of sorting movies based on rate, released date and popularity.\\


Test Name: SS-4 \\
Results: All participants mentioned that all the tasks which they have done above were really easy and so simple at the same time. Their feedback was that our app is easy to install, easy to use, understandable and in overal simple. They were all satisfied but some of them suggested that the software can be improved. For instance one of participants told that software can have an option to sort the movies which  will come out within 6 months. This way people can realize which movies are coming out. Moreover, they enjoyed using the app since it is touchscreen so it has a connection with the user. \\


\subsubsection{Media Output Testing}

The program was installed into phone with an android system operator. The results of attempting to launch the program were noted.\\

Test Name: SS-5\\
Results: The program installed on phones that has android OS and there wasn't any problem.
	
		
\subsection{Performance}
\subsubsection{Screen Speed Performance}
The preformance was calculated based on how long the app takes to perform user requests and if the app run smoothly.\\

Test Name: SS-6 \\
Results: The time to perform user requests was less than 2 second from all participants.

\subsection{Output}
\subsubsection{Media Output Testing}

The time between finding a movie and retrieving the data from dataset was calculated. In addition to that the aim of this test was to check if the output is the expected output. And each time different movie requested by participants to ensure that data, and output file is consistent. Plus, the calculation between sorting movies were noted down in this section.\\ 

Test Name: SS-7 \\
Results: Movies were seatrched by participants retrieved within approximately 3 seconds. \\

Test Name: SS-8 \\
Results: Sorting movies based on rate, popularity, and date released took about 5 seconds and it outputed list of movies based on those categories.

	
\section{Comparison to Existing Implementation}	

This section will not be appropriate for every project.

\section{Unit Testing}

\section{Changes Due to Testing}

\section{Automated Testing}
		
\section{Trace to Requirements}
		
\section{Trace to Modules}		

\section{Code Coverage Metrics}

\bibliographystyle{plainnat}

\bibliography{SRS}

\end{document}